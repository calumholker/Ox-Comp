%
% CO5x report LaTeX example - NTC Nov2011
%

\documentclass[a4paper,11pt]{article}
\usepackage{graphicx}
\usepackage{caption}
\captionsetup{justification=centering,font={small,sf}}
\usepackage{pgfplots}
\pgfplotsset{compat=1.3}
\usepackage{hyperref}
\usepackage{siunitx}
\usepackage{commath}
\usepackage{mathtools}
\usepackage{color}
\usepackage{listings}

\usetikzlibrary{plotmarks}

\definecolor{javared}{rgb}{0.6,0,0} % for strings
\definecolor{javagreen}{rgb}{0.25,0.5,0.35} % comments
\definecolor{javapurple}{rgb}{0.5,0,0.35} % keywords
\definecolor{javadocblue}{rgb}{0.25,0.35,0.75} % javadoc
 
\lstset{language=Java,
basicstyle=\ttfamily,
keywordstyle=\color{javapurple}\bfseries,
stringstyle=\color{javared},
commentstyle=\color{javagreen},
morecomment=[s][\color{javadocblue}]{/**}{*/},
tabsize=4,
showspaces=false,
showstringspaces=false}

\setlength{\voffset}{-25mm}
\setlength{\hoffset}{-10mm}
\setlength{\oddsidemargin}{0mm}
\setlength{\evensidemargin}{0mm}

\setlength{\marginparwidth}{0mm}
\addtolength{\textwidth}{50mm}
\addtolength{\textheight}{55mm}

\begin{document}

\begin{center}
\Large{\textbf{CO5x: Disease Spread}}\\ 
\vspace{1em}
\large{A. N. Student}\\
\large{Some College}
\end{center}

\section{Abstract}

We calculate the spread of disease in two populations, both with and without immunity, 
given known infection and recovery parameters using a simple model for disease spread.
The numbers of people susceptible, infected and recovered from the disease is plotted 
as a function of time. In both populations disease spread is rapid. In the immune case 
infections decrease steadily. Without immunity a steady state of infection is observed to be established.

\section{Introduction}

We assume that the disease starts with a small ($\approx 0.001$) percentage 
of the population, which we categorise as either susceptible to the disease, 
$S(t)$, infected, $I(t)$ or those who have recovered from the disease and 
are now immune (post-infected), $P(t)$. (We are making the assumption that everyone is 
susceptible to the disease at the start of the outbreak and no-one has a 
natural immunity).

The population is closed i.e.\ we have no flow of individuals into or out of the 
population so 
\begin{equation}
   S(t)\ +\ I(t)\ +\ P(t)\ =\ \text{constant}
\end{equation}

To simplify the spread of the disease we assume that when a susceptible person
comes in contact with an infected one there is a (constant) probability $k_1$ of 
the susceptible contracting the disease giving us an expression for the rate of 
change of the susceptible population.
\begin{equation}
\od{S}{t} = -k_1S(t)I(t)
\end{equation}
 
We further assume that there is also a recovery constant $k_2$ which determines 
how quickly an individual recovers from the disease. Thus the number of infected
individuals grows as the number of susceptibles decreases and the infecteds 
decrease as they recover from the disease. This gives us the following formula 
for the rate of change of the infected population.
\begin{equation}
\od{I}{t} = k_1S(t)I(t) - k_2I(t)
\end{equation}

Finally, the number of individuals who have recovered (post-infected) increase
as the number of infected decrease giving us our final equation.
\begin{equation}
\od{P}{t} = k_2I(t)
\end{equation}

The relationships between those susceptible, infected and post-infected are given
by the following differential equations:
\begin{eqnarray}
   \dod{S}{t} &=& -k_1SI         \nonumber \\
   \dod{I}{t} &=& k_1SI - k_2I   \nonumber \\
   \dod{P}{t} &=& k_2I
\label{eqn:model_1}
\end{eqnarray}

The values of $k_1 = \num{e-5}$ and $k_2 = 0.38$ are used for the infected and 
recovery constants.

Our derivation makes the assumption that once a person has recovered from 
the disease then they can not become re-infected. However a rapidly mutating 
disease can overcome the immune system of an individual making then susceptible 
again (like the common cold every winter). To model this in our equations, we 
introduce a new `re-infection' constant $k_3$. The number of individuals 
becoming infected remains the same but the number of susceptibles increase as 
the number of post-infected decrease and vice versa. This modifies our previous 
equations~\ref{eqn:model_1} to
\begin{eqnarray}
\dod{S}{t} &=& -k_1S(t)I(t) +k_3P(t)   \nonumber \\
\dod{I}{t} &=& k_1S(t)I(t) - k_2I(t)   \nonumber \\
\dod{P}{t} &=& k_2I(t) - k_3P(t)
\label{eqn:model_2}
\end{eqnarray}

We use the value of $0.15$ for the re-infection constant $k_3$.

\section{Solution}
The equations given in equation~\ref{eqn:model_1} were solved using the Euler method, which is 
stated as:
\begin{equation}
x_{n+1} = x_n + \Delta t \dod{x}{t}
\end{equation}
where $n$ is the time step such that $t = n\Delta t$. For the first model where the post-infected population is
immune, this method is applied to $S(t)$, $I(t)$ and $P(t)$:
\begin{eqnarray}
S_{n+1} &=& S_n - \Delta t(k_1 S_n I_n)  \nonumber \\
I_{n+1} &=& I_n + \Delta t(k_1 S_n I_n - k_2 I_n)  \nonumber \\
P_{n+1} &=& P_n + \Delta t(k_2 I_n) 
\end{eqnarray}
where $S_0$, $I_0$, and $P_0$ are the initial values. For a total population of one million people, we take $S_0 = 999, 990, I_0 = 10$, and $P_0 = 0$, indicating an initial infection rate of $0.001\%$ and no immunity.

In the second model, where recovered people can fall ill again, the formulae are modified as follows:
\begin{eqnarray}
S_{n+1} &=& S_n + \Delta t(-k_1 S_n I_n + k_3 P_n)  \nonumber \\
I_{n+1} &=& I_n + \Delta t(k_1 S_n I_n - k_2 I_n)  \nonumber \\
P_{n+1} &=& P_n + \Delta t(k_2 I_n - k_3 P_n) 
\end{eqnarray}
The initial conditions are the same as before.

\section{Results}
We solved the equations~\ref{eqn:model_1} to obtain the following graph (figure~\ref{fig:noreinfection})
showing the relationship between susceptible, infected and post-infected people
during an epidemic where post-infected individuals can not be re-infected.
\begin{figure}[h!]
   \centering
\begin{tikzpicture}
\begin{axis}[no markers,xlabel={Day},ylabel={Number infected},width=14cm,height=12cm,legend style={ at={(0.93,0.79)}, anchor=east},scaled y ticks = false,y tick label style={/pgf/number format/fixed,/pgf/number format/1000 sep={}},cycle list={{blue,solid},{green,dashed},{red,dotted}}]
\addplot table [trim cells=true,y=value1]  {epidemic1.dat};
\addlegendentry{Susceptible}
\addplot table [dashed,trim cells=true,y=value2]  {epidemic1.dat};
\addlegendentry{Infected}
\addplot table [dotted,trim cells=true,y=value3]  {epidemic1.dat};
\addlegendentry{Post-infected}
\end{axis}
\end{tikzpicture}
\caption{An epidemic where post-infected individuals can not be re-infected.}
\label{fig:noreinfection}
\end{figure}
We can see the number of susceptible people decrease as they become infected. The number 
of post-infected people always increases as the disease spreads through the population
and everyone has caught and recovered from it.

The solution to the second model (equations~\ref{eqn:model_2}), illustrated in figure~\ref{fig:withreinfection}, shows a marked difference in the
spread of disease. In this case the populations of susceptible, infected and 
post-infected reaches a steady state as those susceptible become infected, recover 
and become susceptible again.
\begin{figure}[h!]
   \centering
\begin{tikzpicture}
\begin{axis}[no markers,xlabel={Day},ylabel={Number infected},/pgf/number format/set thousands separator={},width=14cm,height=12cm,legend style={ at={(0.93,0.79)}, anchor=east},scaled y ticks = false,y tick label style={/pgf/number format/fixed},yticklabels={0,0,1000000,2000000,3000000,4000000,5000000},cycle list={{blue,solid},{green,dashed},{red,dotted}}]
\addplot table [trim cells=true,y=value1]  {epidemic2.dat};
\addlegendentry{Susceptible}
\addplot table [dashed,trim cells=true,y=value2]  {epidemic2.dat};
\addlegendentry{Infected}
\addplot table [dotted,trim cells=true,y=value3]  {epidemic2.dat};
\addlegendentry{Post-infected}
\end{axis}
\end{tikzpicture}
\caption{An epidemic where post-infected individuals can be re-infected.}
\label{fig:withreinfection}
\end{figure}

\section{Conclusion}

The two computer models, the first with immunity and the second without, show the expected behaviour. In both models, because of the small infection rate, the disease progression starts slowly but picks up very rapidly. If the post-infection immunity to the disease is lasting, then the disease decreases steadily. However, if the disease can be caught again (such as through rapid mutation), it can establish itself in a steady fraction of the population. Steady-state infection in fact characterises some common illnesses, such as the common cold.

\clearpage
Nevertheless, the model contains certain simplifying assumptions. For instance:
\begin{enumerate}
\item{} The whole population is susceptible at the outbreak of the disease. This 
        is not the case and so a new parameter can be added to $\od{S}{t}$ to 
        allow for natural immunity. This parameter can decrease with time to allow
        for an immune system being overwhelmed by continued contact with a disease.
\item{} The disease is not fatal. This is not a defect in the model rather a
        consequence of the disease we are looking at.
\item{} The disease carrying organism is long lived. This is not the case in 
        reality where the organisms have a live cycle of their own. The live, 
        reproduce (if conditions are right) and they die. This life-cycle should 
        also be represented in the model which would give us a sensitive dependance
        on the initial parameters of the system. If the reinfection rate is high 
        then there is a constant supply of 'food' for the organism, too low and
        the 'food' runs out as individuals become immune. A similar situation
        will occur if the infection rate is too low. The life time of the
        disease will determine how fast the disease spreads, if the organism
        dies before reaching a host then there is no second generation to 
        carry on infecting the population. So it is possible for the disease
        to either die out or to explode and kill everything by greatly 
        outnumbering the hosts.
\item{} The disease carrying organism is infinitely adaptable. Again this is
        not the case in reality. An `infectibility' parameter should be introduced 
        that determines the probability of an individual becoming infected if
        subjected to the disease. The $k_1$ parameter above is, in reality, an
        transmission parameter determining how the disease is transmitted 
        amongst the population.
\end{enumerate}

Modifying these assumptions would make the model more realistic and could extend it to investigate diseases with different characteristics.

\begin{thebibliography}{9}
\bibitem{cmanual} C.\ W.\ Wiles, A.\ O'Hare et al, \emph{Physics C Programming  Course 2011--2012}, Oxford Physics, 2011\footnote{\href{http://www-teaching.physics.ox.ac.uk/computing/handbook_C.pdf}{\url{http://www-teaching.physics.ox.ac.uk/computing/handbook_C.pdf}}}.
\end{thebibliography}

\clearpage
\appendix
\renewcommand\thesection{Appendix \Alph{section}}

\section{Source Code for Part 1}
The source code for the first problem (where post infected people 
can not be re-infected) is
\begin{lstlisting}
/**********************************************************
 * CO5X: Disease Spread
 * This program calculates the spread of disease in a 
 * population given infections and recovery parameters.
 *
 * "This software uses the gnuplot_i library written by N.Devillard" 
 *
 **********************************************************/
#include <stdio.h>
#include <stdlib.h>
#include <math.h>
#include "gnuplot_i.h"

const double k1 = 0.00001;   /* infection constant */
const double k2 = 0.38;      /* healing constant   */
const double N  = 14.0;      /* days  */

double susceptible(double s, double i)
{
   return -k1*s*i;
}
double infected(double s, double i)
{
   return k1*s*i - k2*i;
}
double immune(double i)
{
   return k2*i;
}

int main()
{
   double S, I, P, t;
   double S_old, I_old;
   double step = 1.0/48.0; /* step measured in days
                          48 half-hour intervals in one day! */

   FILE *fout;
   gnuplot_ctrl *g;

   g = gnuplot_init();

   if ((fout=fopen("epidemic1.dat","w"))==NULL)
   {
      fprintf(stdout, "Cannot open epidemic1.dat\n");
      exit (EXIT_FAILURE);
   }

   S_old = 999990.0;
   I_old = 10.0;
   P = 0.0;
   t = 0.0;

   /* use the Euler method to solve the equations at each point */
   while (t<N)
   {
      S = S_old + step*susceptible(S_old, I_old);
      I = I_old + step*infected(S_old, I_old);
      P = P + step*immune(I_old);
      t = t + step;
      S_old = S;
      I_old = I;
      fprintf(fout, "%f\t%f\t%f\t%f\n", t,S,I,P);
   }
   fclose(fout);

   /* plot the three curves on one graph */
   gnuplot_cmd(g, "plot \'epidemic1.dat\' using 1:2 title 
   				\'Susceptible\' with lines, 
                        \'epidemic1.dat\' using 1:3 title 
                        		\'Infected\' with lines, 
                        \'epidemic1.dat\' using 1:4 title 
                        		\'Post-Infected\' with lines");

   /* Show the graph for 5 seconds before saving it as a postscript file */
   sleep(5);
   gnuplot_cmd(g, "set terminal postscript");
   gnuplot_cmd(g, "set output \"epidemic_part1.ps\"");
   gnuplot_cmd(g, "replot");

   gnuplot_close(g);
   

   return 0;
}
\end{lstlisting}

\section{Source Code for Part 2}
The source code for the second problem (where post infected people 
can be re-infected) is
\begin{lstlisting}
/**********************************************************
 * CO5X: Disease Spread
 * This program calculates the spread of disease in a 
 * population given infections and recovery parameters.
 *
 * "This software uses the gnuplot_i library written by N.Devillard" 
 *
 **********************************************************/
#include <stdio.h>
#include <stdlib.h>
#include <math.h>
#include "gnuplot_i.h"

const double k1 = 0.00001;   /* infection constant */
const double k2 = 0.38;      /* healing constant   */
const double k3 = 0.15;      /* re-infection constant   */
const double N  = 14.0;      /* days  */

double susceptible(double s, double i, double p)
{
   return -k1*s*i + k3*p;
}
double infected(double s, double i)
{
   return k1*s*i - k2*i;
}
double immune(double i, double p)
{
   return k2*i - k3*p;
}

int main()
{
   double S, I, P, t;
   double S_old, I_old, P_old;
   double step = 1.0/48.0; /* step measured in days
                          48 half-hour intervals in one day! */
   FILE *fout;
   gnuplot_ctrl *g;

   g = gnuplot_init();

   if ((fout=fopen("epidemic2.dat","w"))==NULL)
   {
      fprintf(stdout, "Cannot open epidemic2.dat\n");
      exit (EXIT_FAILURE);
   }

   S_old = 999990.0;
   I_old = 10.0;
   P_old = 0.0;
   t = 0.0;

   /* use the Euler method to solve the equations at each point */
   while (t<N)
   {
      S = S_old + step*susceptible(S_old, I_old, P_old);
      I = I_old + step*infected(S_old, I_old);
      P = P_old + step*immune(I_old, P_old);
      t = t + step;
      S_old = S;
      I_old = I;
      P_old = P;
      fprintf(fout, "%f\t%f\t%f\t%f\n", t,S,I,P);
   }
   fclose(fout);

   /* plot the three curves on one graph */
   gnuplot_cmd(g, "plot \'epidemic2.dat\' using 1:2 title 
   					\'Susceptible\' with lines, 
                        \'epidemic2.dat\' using 1:3 title 
                        			\'Infected\' with lines, 
                        \'epidemic2.dat\' using 1:4 title 
                        			\'Post-Infected\' with lines");

   /* Show the graph for 5 seconds before saving it as a postscript file */
   sleep(5);
   gnuplot_cmd(g, "set terminal postscript");
   gnuplot_cmd(g, "set output \"epidemic_part2.ps\"");
   gnuplot_cmd(g, "replot");

   gnuplot_close(g);
   

   return 0;
}
\end{lstlisting}

\section{Makefile for project}
The explanation of the makefile can be found in appendix B of \cite{cmanual}
\begin{lstlisting}
CC 	= gcc
CFLAGS 	= -Wall -O
RM	= rm -f

all:	epidemic_p1 epidemic_p2 

epidemic_p1:	gnuplot_i.o epidemic_p1.o
	$(CC) $(CFLAGS) -o epidemic_p1 epidemic_p1.o gnuplot_i.o -lm

epidemic_p2:	gnuplot_i.o epidemic_p2.o
	$(CC) $(CFLAGS) -o epidemic_p2 epidemic_p2.o gnuplot_i.o -lm

clean:
	$(RM) *.o epidemic_p1 epidemic_p2 *.dat

# compile .c files into .o files
%.o : %.cc
	$(CC) -c $(CFLAGS) $< -o $@
\end{lstlisting}

\end{document}
